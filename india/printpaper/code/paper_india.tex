\documentclass[11pt]{article}
\usepackage{amsmath,amssymb,amsfonts,amsthm}
\usepackage{verbatim}
\usepackage{caption}
\usepackage{graphicx}
\usepackage{natbib}
\usepackage{bibentry}
\usepackage{url}
\usepackage{wrapfig}
\usepackage[margin=1in]{geometry}
\usepackage{framed}
\usepackage{placeins}
\usepackage{bbm}
\usepackage{enumerate}
\usepackage{booktabs}
\usepackage{subfig}   % Supports \subfloat within \figure{} environment
\usepackage{multirow}
\usepackage[setpagesize=false,breaklinks=true]{hyperref}
\hypersetup{
    colorlinks=true,       % false: boxed links; true: colored links
    linkcolor=red,          % color of internal links
    citecolor=black,        % color of links to bibliography
    urlcolor=blue           % color of external links
}

\begin{document}

\section{India}


\begin{figure}
\caption*{Figure 8: India's city-size distribution, urban agglomerations, 2001 and 2011\label{fig:India-UAs-zipf-2001-2011}}
\centering{}\includegraphics[width=0.48\textwidth]{../input/zipfplot_allMSAs.pdf}\includegraphics[width=0.48\textwidth]{../input/zipfplot_2011_UAsandnonUAs.pdf}
\end{figure}

\begin{table}
\caption*{Table 3: India's city-size distribution, subdistrict-night-lights--based metropolitan
areas\label{tab:India-zipf-2001-2011}}
\begin{center}
\input{../input/msa_compare_zipf_India_SDT_2001_2011.tex} \\
\begin{minipage}[t]{0.9\textwidth}%
{\footnotesize \textsc{Notes}:
This table reports the coefficient and $R^{2}$ from a log-linear rank-size regression, as described in the notes of Table 2.
The sample for each regression is a set of Indian metropolitan areas in 2001 or 2011
with population greater than 100,000.
Night-lights--based metropolitan areas are defined by aggregating subdistricts in contiguous areas with light intensity exceeding the listed threshold.\par
}
\end{minipage}
\end{center}
\end{table}


\begin{table} \caption*{Table 8: Population shares for educational categories, 2001} \begin{center}
\input{../input/edu_shares_UA.tex}
\end{center}\end{table}


\begin{table}[ph]
\caption*{Table 9: Population elasticities for educational categories, 2001\label{tab:India-edu-population-elasticities}}
\begin{center}
\input{../input/edu_popelasticity_B9_4scheme_2001_clean.tex}
\begin{minipage}[t]{0.9\textwidth}%
\textsc{\footnotesize{}Notes}{\footnotesize{}:
Each column reports OLS estimates of $\beta_{\nu}$ from a regression defined by equation 1.
Skill fixed effects $\alpha_{\nu}$ are not reported.
Standard errors are clustered by geographic unit.
Each sample contains the union of urban agglomerations and census towns with population greater than 100,000.
Across columns, there is variation in the inclusion threshold,
which is the fraction of the urban agglomerations' population for
which educational attainment data on constituent components is available.\par}%
\end{minipage}
\end{center}
\end{table}

\subsection*{Appendix}

\input{../input/india_sub_districts_area_details.tex}

\begin{table} \caption*{Table C.1: Pairwise comparisons for educational categories} \begin{center}
\input{../input/pairwise_edu_4group_0.tex}
\label{tab:edu_pairwise}
\end{center}\end{table}

\end{document}
