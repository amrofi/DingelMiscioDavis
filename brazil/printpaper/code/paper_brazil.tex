\documentclass[11pt]{article}
\usepackage{amsmath,amssymb,amsfonts,amsthm}
\usepackage{verbatim}
\usepackage{caption}
\usepackage{graphicx}
\usepackage{natbib}
\usepackage{bibentry}
\usepackage{url}
\usepackage{wrapfig}
\usepackage[margin=1in]{geometry}
\usepackage{framed}
\usepackage{placeins}
\usepackage{bbm}
\usepackage{enumerate}
\usepackage{booktabs}
\usepackage{subfig}   % Supports \subfloat within \figure{} environment
\usepackage{multirow}
\usepackage[setpagesize=false,breaklinks=true]{hyperref}
\hypersetup{
    colorlinks=true,       % false: boxed links; true: colored links
    linkcolor=red,          % color of internal links
    citecolor=black,        % color of links to bibliography
    urlcolor=blue           % color of external links
}

\begin{document}

\section{Brazil}

\begin{figure}
\caption*{Figure 3: Comparing population and land area across Brazilian metropolitan-area
definitions, 2010\label{fig:Brazil-correlations-metro-definitions}}
\begin{center}
\includegraphics[width=0.82\textwidth]{../input/msa_duranton_10_baseline_correlation_plot.pdf}
\begin{minipage}{0.8\textwidth}
{\footnotesize
	\textsc{Notes}:
	The left panel depicts correlations of population and land area between metropolitan areas defined by contiguous areas of lights at night using different thresholds.
	The right panel depicts the same for metropolitan areas defined by commuting flows using different thresholds.
	The baseline for comparison in both panels is metropolitan areas defined by commuting flows in 2010 with a 10\% threshold, per \cite{Duranton:2015}.
	Thus, the perfect correlation at the 10\% threshold in the right panel is tautological.
	Footnote 8 describes how we pair metropolitan areas with their baseline counterparts.
	The sample is restricted to metropolitan areas with population above 100,000.\par
}
\end{minipage}
\end{center}

\end{figure}


\begin{table} \caption*{Table 4: Population shares for educational categories, 2010} \begin{center}
\input{../input/edu_shares_Census_All_and_Metro.tex}
\end{center}\end{table}


\begin{table}[pth]
\caption*{Table 5: Population elasticities for educational categories, 2010\label{tab:Brazil-edu-Population-elasticities}}
\begin{centering}
\resizebox{\textwidth}{!}{\input{../input/edupop_elasticities_Census2010_msa_compare_clean.tex}}
\par\end{centering}
\centering{}%
\begin{minipage}[t]{0.95\textwidth}%
\textsc{\footnotesize{}Notes}{\footnotesize{}:
Each column reports OLS estimates of $\beta_{\nu}$ from a regression defined by equation 1.
Skill fixed effects $\alpha_{\nu}$ are not reported.
Standard errors are clustered by geographic unit.
Each sample contains geographic units with population greater than 100,000.\par}
\end{minipage}
\end{table}


\begin{figure}[pbh]
\caption*{Figure 9: Non-parametric population elasticities for educational categories,
2010\label{fig:Brazil-edu-Non-parametric}}
\begin{centering}
\includegraphics[width=0.75\textwidth]{../input/popelast_edu_lpoly.pdf}
\par\end{centering}
\centering{}%
\begin{minipage}[t]{0.8\textwidth}%
\textsc{\footnotesize{}Notes}{\footnotesize{}:
Each series plots a local mean smoother using an Epanechnikov kernel.
Metropolitan areas are defined by commuting ties between municipios, using Duranton (2015) algorithm with 10\% threshold.
The histogram bars depict the number of metropolitan areas (on the right vertical axis).\par}%
\end{minipage}
\end{figure}


\begin{table}[pht]
\caption*{Table 10: Skill gradient in Brazilian metropolitan areas, 2010}
\label{tab:skillgradient}
\begin{center}
\input{../input/distance_gradient_college.tex}
\begin{minipage}{0.92\textwidth}
{\footnotesize \textsc{Notes}:
	The dependent variable is the share of residents who are college graduates in a municipio.
	Distance to metro center is measured from the municipio centroid to the population-weighted average of constituent-municipio centroids as a share of the greatest distance.
	The sample is restricted to metropolitan areas containing at least two municipios.
	Standard errors, clustered by metropolitan area, are in parentheses.
	The reported $p$-values test the null hypotheses that the coefficients estimated when using night-lights--based metropolitan areas are equal to the coefficients estimated when using arranjos or microregions.
	The table also reports the number of SUR clusters used to compute those test statistics;
	see Appendix C.2 for details.\par
}
\end{minipage}
\end{center}
\end{table}


\begin{figure}[phb]
\caption*{Figure 10: Skill gradient in Brazilian metropolitan areas, 2010}
\label{fig:skillgradient30}
\centering{} \includegraphics[width=0.6\textwidth]{../input/distance_gradient_college_msa_night_30.pdf}
\begin{minipage}{0.8\textwidth}
{\footnotesize \textsc{Notes}:
	The two series are local mean smoothers of the share of residents who are college graduates in a municipio as a function of distance to metro center using an Epanechnikov kernel.
	Dashed lines depict 95\% confidence intervals.
	Metropolitan areas are defined by aggregating municipios using a light-intensity threshold of 30.\par
}
\end{minipage}
\end{figure}


\begin{table}
\caption*{Table 12: Average nominal wages across Brazilian metropolitan areas, 2010}
\label{tab:urbanwagepremia:Brazil}
\begin{center}
\input{../input/hce_pop_wage_Census2010.tex}
\begin{minipage}{0.93\textwidth}
{\footnotesize
	\textsc{Notes}:
	The dependent variable is the average nominal hourly wage in a metropolitan-area $\times$ gender $\times$ age $\times$ race $\times$ education cell.
	The college graduate share takes values between 0 and 100.
	Unreported controls are fixed effects for gender, age, race, and educational attainment.
	Standard errors, clustered by metropolitan area, in parentheses.
}
\end{minipage}
\end{center}
\end{table}


\begin{table}
\caption*{Table 13: Skill premia in Brazilian metropolitan areas, 2010}
\label{tab:skillpremia:Brazil}
\begin{center}
\input{../input/skillpremia3_Census2010.tex}
\begin{minipage}{0.9\textwidth}
{\footnotesize
	\textsc{Notes}:
	The dependent variable is a metropolitan area's difference in average log hourly wages between college graduates and high school graduates.\par
}
\end{minipage}
\end{center}
\end{table}



\begin{figure}[hp]
\caption*{Figure 11: Brazil occupational employment population elasticities, 2010\label{fig:Brazil-occ-elasticities}}
\begin{centering}
\includegraphics[width=0.45\textwidth]{../input/occpop_elasticities_Census2010}\includegraphics[width=0.45\textwidth]{../input/occpop_elasticities_Census2010_zoom}
\par\end{centering}
\centering{}%
\begin{minipage}[t]{0.85\textwidth}%
\textsc{\footnotesize{}Notes}{\footnotesize{}:
Each observation is an occupational category.
The population elasticity of employment is estimated by linear regression.
Skill intensity is the average years of schooling of persons employed in that occupational category.
Bubble sizes are proportionate to the occupational category's share of employment.
Metropolitan areas are defined by commuting flows between municipios, using the Duranton (2015) algorithm with a 10\% threshold.
Left panel includes all occupations; right panel omits agriculture.\par}%
\end{minipage}
\end{figure}


\begin{figure}[hp]
\caption{Fgiure 12: Brazil industrial employment population elasticities, 2010\label{fig:Brazil-ind-elasticities}}
\begin{centering}
\includegraphics[width=0.45\textwidth]{../input/indpop_elasticities_Census2010}\includegraphics[width=0.45\textwidth]{../input/indpop_elasticities_Census2010_zoom}
\par\end{centering}
\centering{}%
\begin{minipage}[t]{0.85\textwidth}%
\textsc{\footnotesize{}Notes}{\footnotesize{}:
Each observation is an industrial category.
The population elasticity of employment is estimated by linear regression.
Skill intensity is the average years of schooling of persons employed in that industrial category.
Bubble sizes are proportionate to the industrial category's share of employment.
Metropolitan areas defined by commuting ties between municipios, using Duranton (2015) algorithm with 10\% threshold.\par}%
\end{minipage}
\end{figure}

\subsection*{Appendix}

\input{../input/brazil_municipios_area_details.tex}

\begin{table} \caption*{Table C.1: Pairwise comparisons for educational categories, 2010} \begin{center}
\input{../input/pairwise_edu.tex}
\label{tab:edu_pairwise}
\end{center}\end{table}

\end{document}
